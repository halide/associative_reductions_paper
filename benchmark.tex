\lstset{basicstyle=\fontsize{9pt}{10pt}\ttfamily}

\begin{minipage}{\linewidth}
\begin{lstlisting}[caption={Histogram of a two-dimensional image.}, label={lst:benchmark_histogram}]
Func hist;
Var x, y;
RDom r(0, size, 0, size);

hist(x) = 0;
hist(in(r.x, r.y)) += 1;

RVar ryo, ryi;
hist
  .update()
  .split(r.y, ryo, ryi, 16)
  .rfactor(ryo, y)
  .compute_root()
  .vectorize(x, 8)
  .update().parallel(y);
\end{lstlisting}
\end{minipage}

\begin{minipage}{\linewidth}
\begin{lstlisting}[caption={Maximum value over a 1D array}, label={lst:benchmark_max}]
Func maxf;
RDom r(0, size);

maxf() = 0;
maxf() = max(maxf(), A(r));

RVar rxo, rxi, rxio, rxii;
Var u, v;
Func intm =
  maxf.update()
      .split(r.x, rxo, rxi, 4*8192)
      .rfactor(rxo, u);
intm.compute_root()
    .update()
    .parallel(u)
    .split(rxi, rxio, rxii, 8)
    .rfactor(rxii, v)
    .compute_at(intm, u)
    .vectorize(v)
    .update()
    .vectorize(v);
\end{lstlisting}
\end{minipage}

\begin{minipage}{\linewidth}
\begin{lstlisting}[caption={Argmin over 4D array}, label={lst:benchmark_argmin}]
Func amin;
RDom r(0, size, 0, size, 0, size, 0, size);

amin() = {255, 0, 0, 0, 0};
amin() = {
  min(amin()[0], input(r.x, r.y, r.z, r.w)),
  select(amin()[0] < input(r.x, r.y, r.z, r.w),
         amin()[1], r.x),
  select(amin()[0] < input(r.x, r.y, r.z, r.w),
         amin()[2], r.y),
  select(amin()[0] < input(r.x, r.y, r.z, r.w),
         amin()[3], r.z),
  select(amin()[0] < input(r.x, r.y, r.z, r.w),
         amin()[4], r.w)
};

Var u, v;
Func intm1 = amin.update().rfactor(r.w, u);
intm1.compute_root().update().parallel(u);
RVar rxo, rxi;
Func intm2 =
  intm1.update()
       .split(r.x, rxo, rxi, 16)
       .rfactor(rxi, v);
intm2.compute_at(intm1, u)
     .update().vectorize(v);
\end{lstlisting}
\end{minipage}

\begin{minipage}{\linewidth}
\begin{lstlisting}[caption={Complex product}, label={lst:benchmark_complex_multiply}]
Func mult;
RDom r(0, size);

mult() = {1, 0};
mult() = {
 mult()[0]*input0(r) - mult()[1]*input1(r),
 mult()[0]*input1(r) + mult()[1]*input0(r)
};

RVar rxi, rxo, rxii, rxio;
Var u, v;
Func intm =
  mult.update()
      .split(r.x, rxo, rxi, 2*8192)
      .rfactor(rxo, u);
intm.compute_root()
    .vectorize(u, 8)
    .update()
    .parallel(u)
    .split(rxi, rxio, rxii, 8)
    .rfactor(rxii, v)
    .compute_at(intm, u)
    .vectorize(v)
    .update()
    .vectorize(v);
\end{lstlisting}
\end{minipage}

\begin{minipage}{\linewidth}
\begin{lstlisting}[caption={Dot product}, label={lst:benchmark_dot_product}]
Func dot;
RDom r(0, size);

dot() = 0;
dot() += cast<int>(A(r.x))*B(r.x);

RVar rxo, rxi, rxio, rxii;
Var u, v;
Func intm =
  dot.update()
     .split(r.x, rxo, rxi, 4*8192)
     .rfactor(rxo, u);
intm.compute_root()
    .update()
    .parallel(u)
    .split(rxi, rxio, rxii, 8)
    .rfactor(rxii, v)
    .compute_at(intm, u)
    .vectorize(v)
    .update()
    .vectorize(v);
\end{lstlisting}
\end{minipage}

\begin{minipage}{\linewidth}
\begin{lstlisting}[caption={Everything but the kitchen sink.}, label={lst:benchmark_kitchen_sink}]
Func sink;
RDom r(0, size);

sink() = {
  0, 0, int(0x80000000), 0,
  int(0x7fffffff), 0, 0, 0
};
sink() = {
  sink()[0] * A(r),
  sink()[1] + A(r),
  max(sink()[2], A(r)),
  select(sink()[2] > A(r), sink()[3], r),
  min(sink()[4], A(r)),
  select(sink()[4] < A(r), sink()[5], r),
  sink()[6] + A(r)*A(r),
  sink()[7] + select(A(r) % 2 == 0, 1, 0)
};

RVar rxo, rxi, rxio, rxii;
Var u, v;
Func intm =
  sink.update()
      .split(r.x, rxo, rxi, 8192)
      .rfactor(rxo, u);
intm.compute_root()
    .update()
    .parallel(u)
    .split(rxi, rxio, rxii, 8)
    .rfactor(rxii, v)
    .compute_at(intm, u)
    .vectorize(v)
    .update()
    .vectorize(v);
\end{lstlisting}
\end{minipage}
