\subsection{Reductions in Halide}

Serial reduction operations in Halide (e.g. summation over an array, histogram, etc.) is implemented through usage of non-data-parallel \code{RVar}s. Since \code{RVar}s are not readily parallelizable or vectorizable, programmer needs to manually \emph{factorize} the reduction stage into an \emph{intermediate} stage that performs reduction over some partial domains, and a \emph{merge} stage that combines those partial results. This whole manipulation process can be pretty tedious and error-prone to do manually, especially when the reduction domain is non-rectangular -- In Halide, programmer can specify predicates on a reduction domain via the \code{where} directive (see Listing \ref{lst:circular_sum}). To further complicate matters, some reduction operations are not obviously associative, e.g. \code{x + y + k*x*y} in which \code{k} is an integer constant.

To remove the burden of \emph{factorizing} a reduction from the programmer, we introduce a new scheduling primitive called \code{rfactor} in Halide. \code{rfactor} makes a best-effort attempt to automatically synthesize an equivalent binary associative reduction operator and its identity from a serial Halide reduction, and uses them to automatically generate the \emph{intermediate} and \emph{merge} stages.

\subsection{\code{rfactor} DAG Transformation}

Throughout this section, we will assume that the equivalent binary associative reduction operator of a Halide reduction and its identity are known. We present a Halide scheduling primitive, \code{rfactor}, that splits a reduction stage into pair of stages: \emph{intermediate} and \emph{merge} stages. \code{rfactor} takes as input \code{preserved}, which is a list of \code{<RVar, Var>} pairs. \code{RVar}s not in \code{preserved} are removed from the \emph{merge} stage and are lifted to the \emph{intermediate} stage. The remaining \code{RVar} (the ones in \code{preserved}) are made into \code{Var}s in the \emph{intermediate} stage, which allows them to be parallelized or vectorized. Listing \ref{lst:histogram_rfactor} and \ref{lst:circular_sum} demonstrate usage of \code{rfactor} to parallelize histogram of two-dimensional image and summation over a circular reduction domain with radius of 10. 

It is worth noting that we limit the scope of \code{rfactor} to reductions where the \emph{intermediate} and \emph{merge} stages have the same equivalent binary associative reduction operator; for instance, if the equivalent binary associative reduction operator of the \emph{intermediate} stage is \code{min(x, y)}, then that of the \emph{merge} stage must also be \code{min(x, y)}. Another restriction is that the binary associative operator must have an identity as it is used to initialize the \emph{intermediate} stage.

<TODO: Insert rfactor for matrix multiply> \\

\begin{lstlisting}[caption={Histogram of a two-dimensional image: serial vs. parallel hand-rolled vs. parallel rfactor}, label={lst:histogram_rfactor}]
// Serial version
Func hist;
Var x("x");
hist(x) = 0;
RDom r(input.width(), input.height());
hist(clamp(cast<int>(input(r.x, r.y)), 0, 255)) =
  hist(clamp(cast<int>(input(r.x, r.y)), 0, 255))
  + 1;

// Hand-rolled version: compute the histogram 
// along the x scanline for each y in parallel, 
// then reduce the partial results
Func intm;
Var u("u");
intm(u) = 0;
RDom rx(input.width());
intm(clamp(cast<int>(input(rx, u)), 0, 255)) =
  intm(clamp(cast<int>(input(rx, u)), 0, 255)) + 1;
intm.compute_root();
intm.update(0).parallel(u);

Func hist;
Var x("x");
hist(x) = 0;
RDom ry(input.height());
hist(clamp(cast<int>(intm(ry)), 0, 255)) =
  hist(clamp(cast<int>(intm(ry)), 0, 255)) + 1;

// rfactor version: compute the histogram 
// along the x-scanline for each y in parallel, 
// then reduce the partial results
Func hist;
Var x("x");
hist(x) = 0;
RDom r(input.width(), input.height());
hist(clamp(cast<int>(input(r.x, r.y)), 0, 255)) =
  hist(clamp(cast<int>(input(r.x, r.y)), 0, 255)) 
  + 1;

Var u("u");
Func intm = hist.update(0).rfactor(r.y, u);
intm.compute_root();
intm.update(0).parallel(u);
\end{lstlisting}

\begin{lstlisting}[caption={Sum of squares reduction over a circular reduction domain with radius of 10: serial vs. parallel hand-rolled vs. parallel rfactor}, label={lst:circular_sum}]
// Serial version
Func sum;
sum() = 0;
RDom r(input.width(), input.height());
r.where(r.x*r.x + r.y*r.y <= 100);
sum() = sum() + input(r.x, r.y)*input(r.x, r.y);

// Hand-rolled version: compute the sum of  
// squares along the x scanline for each y in 
// parallel, then reduce the partial results
Func intm;
Var u("u");
intm(u) = 0;
RDom rx(input.width());
rx.where(rx*rx + u*u <= 100);
intm(u) = intm(u) + input(rx, u)*input(rx, u);
intm.compute_root();
intm.update(0).parallel(u);

Func sum;
Var x("x");
sum() = 0;
RDom ry(input.height());
sum() = sum() + intm(ry);

// rfactor version : compute the sum of 
// squares along the x scanline for each y in  
// parallel, then reduce the partial results
Func sum;
Var x("x");
sum() = 0;
RDom r(input.width(), input.height());
r.where(r.x*r.x + r.y*r.y <= 100);
sum() = sum() + input(r.x, r.y)*input(r.x, r.y);

Var u("u");
Func intm = sum.update(0).rfactor(r.y, u);
intm.compute_root();
intm.update(0).parallel(u);
\end{lstlisting}
