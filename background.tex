Enough information about Halide. \\

How does existing work not quite solve it? \\

Work on synthesizing parallel (data parallelism + vectorization) reductions out of serial code. \\

Work on deducing associative operators: \\
\begin{enumerate}
	\item Deducing the map/reduce operators via theory of list homomorphisms + weak right inverse by Morita et al. Con: User have to specify the leftwards and rightwards forms of the sequential function which may not be obvious to derive. 
	\item Using induction to derive the parallel form by Chin et al. Steps: derive 2 pre-parallel funcs through generalization and deduce the "unknowns" (initial/final reduction functions) through induction of those two funcs. Seems promising -> can solve for intermediate/combiner for complex multiplication, albeit through long derivation. Argmin is tricky though: need to be able to re-order chain of selects, which is not trivial, during the induction steps, e.g. transforming select(min(f(x0)[0], g(y)) < g(x), select(f(x)[0] < g(y), f(xs)[1], y), x) into select(f(xs)[0] < min(g(x), g(y)), f(xs)[1], select(g(x) < g(y), x, y)). 	 
	\item Deducing the map/reduce operators via (backward) synthesis by Smith et al. Not sure how the whole pipeline actually works. They claims they can synthesize the right map/reduce ops given some input/output examples. Table 2 listed the components they used. I would assume they search for combination of components until they get the right one. The time it takes to synthesize the map/reduce seems to be okay, but they said they used only small inputs (3-8 samples) to get the benchmark data. \\
\end{enumerate}
	
We are inspired by synthesis and tried "Sketch" (by Solar-Lezama) and "Rosette" (by Torlak et al) to synthesize initial/final reduction functions for the rfactor of 32-bit integer simplex complex multiplication. "Sketch" failed to find one within reasonable time; Rosette takes about one hour (with some constraints: x are fixed only to appear in RHS and depth is set to 1). Backward synthesis -> really slow, which inspires us to do forward synthesis.\\
	
Work on superoptimization (in particular, Regehr et al and Mangpo's ASPLOS16 paper, but also earlier work)

