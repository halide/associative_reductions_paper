Programmers define the \emph{algorithm} in Halide through a Halide \emph{function}, which consists of sequence of stages; these stages are the unit on which scheduling occurs. By default, each stage represents a perfectly-nested loop nest in which a single value of the \emph{function} is computed and stored in the innermost loop per iteration. Stages after the first are called \emph{update} stages, and are allowed to recursively refer to the function. Some of the loops are data parallel and are constrained to be race-condition free by syntactic restrictions. These data-parallel loops iterate over variables called \code{Var}s. The bounds of these loops are inferred by Halide using interval arithmetic.  Other loops may have user-specified bounds and a user-specified nesting order, and fewer syntactic restrictions on their use. These are known as \code{RVar}s (for reduction \code{Vars}), which together define a reduction domain or \code{RDom}. \code{RVar}s are used to express reductions, scattering, scans, etc. Each of these loop types, defined by \code{Var}s and \code{RVar}s, can be manipulated in various ways through Halide scheduling primitives: they can be tiled, unrolled, mutually interchanged, etc., provided that the nesting order of \code{RVar}s is respected. 

While \code{Var}s are safe to parallelize or vectorize by construction -- \code{Var}s represents the naturally data-parallel axes of an \emph{algorithm} -- \code{RVar}s can be parallelized or vectorized if and only if Halide can prove that no race condition exists. This makes parallelizing or vectorizing stages that use only \code{RVar}s difficult. For example, consider the two-dimensional convolutional blur kernel shown in Listing~\ref{lst:blur_loopness}, which is easily parallelizable across \code{Var} $x$ and $y$. The histogram of an image (see Listing~\ref{lst:histogram_loopness}), on the other hand, is harder to parallelize since its update stage only involves \code{RVar}s. In order to parallelize a reduction like histogram, one needs to be able to factorize it into slices that have no dependencies on each other.

Although much prior work has explored automatic generation of parallel associative reductions from a serial reduction, most work assumes an explicit associative binary reduction operator is given, which is not applicable to Halide. Since Halide does not support reduction using a binary operator as a first-class primitive, reductions in Halide are implemented through usage of non-data-parallel \code{RVar}s. For Halide to support parallel reductions, it needs to be able to deduce an equivalent binary associative reduction operator and its identity from a serial reduction expressed as an imperative Halide \emph{update}. 

Prior work by Morita et al.~\cite{Morita:2007:AIG:1250734.1250752} introduced automatic generations of divide-and-conquer parallel programs framework based on the third homomorphism theorem and derivation of weak-right inverse. However, it requires programmers to specify the leftwards and rightwards forms of the sequential function which may not be obvious to derive. Teo et al.~\cite{Teo:1997:DEP:266670.266697} proposed a method to synthesize parallel divide-and-conquer
programs from a recurrence function (which is similar in form to a Halide serial reduction) through induction. They first derive two equivalent pre-parallel forms of the recurrence function by applying some generalization rules and deduce the intermediate and merge reduction functions through induction on those two pre-parallel forms. Although it can be applied to solve some complex recurrences, such as reduction of complex multiplication, the technique requires long derivation time and is unable to deal with reductions like \code{argmin}, which require non-trivial re-ordering of the chain of conditionals during the induction steps. 

Recent work has applied program synthesis, which automatically discovers executable code based on user intent derived from examples or other constraints, to generate parallel programs. Smith et al.~\cite{Smith:2016:MPS:2908080.2908102} used program synthesis to automatically generate MapReduce-style distributed programs from input-output examples. \textsc{Sketch}~\cite{Solar-Lezama:2008:PSS:1714168} and \textsc{Rosette}~\cite{Torlak:2013:GSL:2509578.2509586} are two solver-aided programming languages with support for program synthesis.  MSL~\cite{Xu:2014:MSE:2683593.2683628} is a synthesis-based language for distributed implementations that can derive many details of the distributed implementation from serial specifications.  We discuss \textsc{Sketch} and \textsc{Rosette} in Section~\ref{synthesize}.

Superoptimization~\cite{Granlund:1992:EBU:143095.143146, Massalin:1987:SLS:36206.36194} searches for the shortest or most optimized way to compute a branch-free sequence of instructions, by exhaustively searching over a space of possible programs. These rewrites can then be turned into peephole optimizations in compilers. More recent work has used stochastic search~\cite{Phothilimthana:2016:SUS:2872362.2872387, Schkufza:2013:SS:2490301.2451150} and program synthesis~\cite{Lopes:2015:PCP:2737924.2737965} to find replacements for larger sequences of instructions.
In this work, we find equivalent replacements for a Halide reduction through a combination of enumeration and synthesis; in addition, though our domain is more restricted, we search for larger replacements than most superoptimizers.

\begin{lstlisting}[caption={Convolution blur kernel is easily parallelizable across \code{Var} $x$ adn $y$.}, label={lst:blur_loopness}]
// First stage
for (int y = 0; y < input.height(); ++y) {
  for (int x = 0; x < input.width(); ++x) {
    blur[x][y] = 0;
  }
}
// Update stage
parallel for (int y = 0; y < input.height(); ++y) {
  parallel for (int x = 0; x < input.width(); ++x) {
    for (int ry = 0; ry < kernel.height(); ++ry) {
      for (int rx = 0; rx < kernel.width(); ++rx) {
        blur[x][y] += 
          kernel[rx][ry]*input[x+rx-1][y+ry-1];
      }
    }
  }
} 
\end{lstlisting}

\begin{lstlisting}[caption={Histogram of an image is hard to parallelize since its update stage does not involve \code{Var}s.}, label={lst:histogram_loopness}]
// Serial version
// First stage
for (int x = 0; x < 256; ++x) {
  hist[x] = 0;
}
// Update stage
for (ry = 0; ry < input.height(); ++ry) {
  for (rx = 0; rx < input.width(); ++rx) {
    hist[clamp(int(input[rx][ry]), 0, 255)] += 1;
  }
}
\end{lstlisting}
