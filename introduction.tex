Halide \cite{Ragan-Kelley:2013:HLC:2491956.2462176} is a domain-specific language designed for fast-image processing and computational photography. Halide decouples the \emph{algorithm}, which defines \emph{what} values are computed, from the \emph{schedule}, which defines \emph{how} values are computed. Halide guarantees consistency -- an algorithm produces the same results no matter the schedule. Programmers are thus free to explore the space of schedules without introducing correctness bugs, and can vary the schedule per architecture without producing different results on different platforms.

Data-parallel operations, such as resizing an image, can be easily parallelized or vectorized in Halide. However, Halide does not support first-class parallel reductions. To parallelize or vectorize a reduction, the programmer has to manually factor the reduction into multiple stages to expose new data parallelism. For example, instead of computing the histogram of an entire image, one might instead write an algorithm that computes the histogram of each row, and then adds those partial histograms. This violates the core tenet of Halide: The algorithm should only specify \emph{what} is computed. It is the role of the schedule to specify \emph{how}. This manipulation of the \emph{algorithm} to parallelize reductions is hence bug-prone and hampers readability and portability. It is a language wart.

We present a new Halide scheduling primitive called \code{rfactor}, which moves this factoring of a reduction into the \emph{schedule}, while maintaining Halide's consistency guarantees. \code{rfactor} takes a Halide serial reduction (expressed with an unstructured Halide "update" definition), and synthesizes the equivalent binary associative reduction operator and its identity. In some cases this synthesis problem is trivial. For example, given Halide code that sums a one-dimensional vector (Listing \ref{lst:sum}), it is straight-forward to deduce that the binary operator involved is addition, and its identity is zero. In other cases this synthesis problem is more challenging. Listing \ref{lst:complex_magnitude} shows Halide code that finds the complex number with the greatest magnitude and its location within a two-dimensional array. It is not obvious what the equivalent associative binary operator is for this algorithm.

During the compilation process, \code{rfactor} splits the original serial reduction into pair of stages: The \emph{intermediate} stage computes partial results over slices of the domain of the reduction, and the \emph{merge} stage combines those partial results. The intermediate stage is now data parallel over the slices, which means that it can now be vectorized or parallelized using Halide's existing scheduling primitives.

Combined with other Halide scheduling primitives, such as \code{split}, \code{rfactor} allows Halide to represent a broad class of schedules for parallel reductions. For example, one might schedule a parallel divide-and-conquer summation over a one-dimensional array by first splitting the array reduction dimension into inner and outer subdimensions using \code{split}, then using \code{rfactor} to compute the partial sums over the inner subarrays independently, which then allows parallelizing over the outer subdimension (See Listing \ref{lst:sum}). A second application of \code{split} and \code{rfactor} could vectorize the summation within each subarray, adding a second intermediate stage which sums over vector lanes.

\code{rfactor} further separates the \emph{algorithm} from its \emph{schedule} by making it possible to factor reductions using the schedule alone. This means that tools that automatically generate schedules (\cite{Mullapudi:2016:ASH:2897824.2925952}\cite{Ragan-Kelley:2013:HLC:2491956.2462176}) are now capable of parallelizing reductions, which was previously a task outside of their purview.

%Our work makes the following contributions:
%\begin{itemize}
%  \item We introduce a new Halide scheduling primitive \code{rfactor} which transforms a Halide reduction stage into pair of stages: \emph{intermediate} stage that computes partial results over slices of reduction domain and \emph{merge} stage that combines those partial results;
%  \item We descibe a method for automatic synthesis of an equivalent associative binary reduction operator and its identity from a serial reduction expressed as an imperative Halide ``update''.
%\end{itemize}

The paper is structured as follows. Section \ref{background} provides background on Halide and a discussion of related work. Section \ref{assoc_red} presents the \code{rfactor} scheduling primitive and how it transforms Halide programs. Section \ref{synthesis} describes the associative binary reduction operator synthesis technique. Section \ref{evaluation} describes limitations of the technique, and demonstrates that this technique does indeed produce the expected performance gains from vectorization and parallelization.

\begin{lstlisting}[
caption = {Halide sum reduction over a one-dimensional vector}, label={lst:sum}]
Func out;
out() = 0;
RDom r(0, input.width());
out() = out() + input(r.x);
\end{lstlisting}

\begin{lstlisting}[
caption = {Halide reduction which finds the complex number with the greatest magnitude and its location in a two-dimensional array}, label={lst:complex_magnitude}]
Func out;
out() = {0, 0, 0, 0};
RDom r(0, input.width(), 0, input.height());
Expr real = input(r.x, r.y)[0];
Expr imag = input(r.x, r.y)[1];
Expr mag = real * real + imag * imag;
Expr best_mag = out()[0] * out()[0] +
                          out()[1] * out()[1];
Expr c = mag > best_mag;
out() = {select(c, real, out()[0]),
             select(c, imag, out()[1]),
             select(c, r.x, out()[2]),
             select(c, r.y, out()[3])};
\end{lstlisting}
