% This is "sig-alternate.tex" V2.1 April 2013
% This file should be compiled with V2.5 of "sig-alternate.cls" May 2012
%
% This example file demonstrates the use of the 'sig-alternate.cls'
% V2.5 LaTeX2e document class file. It is for those submitting
% articles to ACM Conference Proceedings WHO DO NOT WISH TO
% STRICTLY ADHERE TO THE SIGS (PUBS-BOARD-ENDORSED) STYLE.
% The 'sig-alternate.cls' file will produce a similar-looking,
% albeit, 'tighter' paper resulting in, invariably, fewer pages.
%
% ----------------------------------------------------------------------------------------------------------------
% This .tex file (and associated .cls V2.5) produces:
%       1) The Permission Statement
%       2) The Conference (location) Info information
%       3) The Copyright Line with ACM data
%       4) NO page numbers
%
% as against the acm_proc_article-sp.cls file which
% DOES NOT produce 1) thru' 3) above.
%
% Using 'sig-alternate.cls' you have control, however, from within
% the source .tex file, over both the CopyrightYear
% (defaulted to 200X) and the ACM Copyright Data
% (defaulted to X-XXXXX-XX-X/XX/XX).
% e.g.
% \CopyrightYear{2007} will cause 2007 to appear in the copyright line.
% \crdata{0-12345-67-8/90/12} will cause 0-12345-67-8/90/12 to appear in the copyright line.
%
% ---------------------------------------------------------------------------------------------------------------
% This .tex source is an example which *does* use
% the .bib file (from which the .bbl file % is produced).
% REMEMBER HOWEVER: After having produced the .bbl file,
% and prior to final submission, you *NEED* to 'insert'
% your .bbl file into your source .tex file so as to provide
% ONE 'self-contained' source file.
%
% ================= IF YOU HAVE QUESTIONS =======================
% Questions regarding the SIGS styles, SIGS policies and
% procedures, Conferences etc. should be sent to
% Adrienne Griscti (griscti@acm.org)
%
% Technical questions _only_ to
% Gerald Murray (murray@hq.acm.org)
% ===============================================================
%
% For tracking purposes - this is V2.0 - May 2012

\documentclass{main}
\pagenumbering{arabic}

\usepackage{listings}

\begin{document}

% Copyright
%\setcopyright{acmcopyright}
%\setcopyright{acmlicensed}
%\setcopyright{rightsretained}
%\setcopyright{usgov}
%\setcopyright{usgovmixed}
%\setcopyright{cagov}
%\setcopyright{cagovmixed}


\title{Parallel Associative Reductions in Halide}

%\numberofauthors{3}
%\author{
%% 1st. author
%\alignauthor
%Ben Trovato\titlenote{Dr.~Trovato insisted his name be first.}\\
%       \affaddr{Institute for Clarity in Documentation}\\
%       \affaddr{1932 Wallamaloo Lane}\\
%       \affaddr{Wallamaloo, New Zealand}\\
%       \email{trovato@corporation.com}
%% 2nd. author
%\alignauthor
%G.K.M. Tobin\titlenote{The secretary disavows
%any knowledge of this author's actions.}\\
%       \affaddr{Institute for Clarity in Documentation}\\
%       \affaddr{P.O. Box 1212}\\
%       \affaddr{Dublin, Ohio 43017-6221}\\
%       \email{webmaster@marysville-ohio.com}
%% 3rd. author
%\alignauthor Lars Th{\o}rv{\"a}ld\titlenote{This author is the
%one who did all the really hard work.}\\
%       \affaddr{The Th{\o}rv{\"a}ld Group}\\
%       \affaddr{1 Th{\o}rv{\"a}ld Circle}\\
%       \affaddr{Hekla, Iceland}\\
%       \email{larst@affiliation.org}
%}

\maketitle
\begin{abstract}

Halide is a domain-specific language for fast image processing in heavy use in industry. It partitions pipelines into the \emph{algorithm}, which defines \emph{what} values are computed, and the \emph{schedule}, which defines \emph{how} they are computed. While Halide supports parallelizing and vectorizing naturally data-parallel operations, such as resizing an image, it doesn't have first-class parallel reductions. To parallelize or vectorize a reduction, the programmer must manually factor a reduction into multiple stages to create new opportunities for data parallelism. For example, one might parallelize the computation of the histogram of an image by first computing the histogram of each row in parallel, and then summing those partial histograms. This manipulation of the \emph{algorithm} can introduce bugs, make the algorithm harder to understand, and hurts portability. We describe a new Halide scheduling primitive ``rfactor'' which moves this factoring into the \emph{schedule}, and provides the same guarantees as the rest of the scheduling language: changes to the schedule can't change the output values or introduce race conditions. Our technique take serial Halide reductions (expressed using unstructured ``update'' definitions), and synthesizes an equivalent binary associative reduction operator and its identity. During compilation, we replace the original pipeline stage with a pair of stages that first computes partial results over slices of the domain of the reduction, and then combines them. Our technique permits parallelization and vectorization of Halide algorithms for which this was not previously possible, and can thus accelerate them by more than an order of magnitude.

\end{abstract}


%
%  Use this command to print the description
%
\printccsdesc

%\keywords{ACM proceedings; \LaTeX; text tagging}

\section{Introduction}
Halide \cite{Ragan-Kelley:2013:HLC:2491956.2462176} is important, provides separation of algorithm and schedule. Ability to try various schedules with guaranteed correctness and consistency: different schedules are guaranteed to produce the same output as long they define the same computation. \\

One important class of problems: associative reduction. This can be as simple as single-dimensional sum or some complex multidimensional associative ops (see Figure 1). One way to optimize performance in associative reduction: split into smaller chunks of works, compute them separately, and merge the partial result. Associative property allows such optimization. \\

Figure:

\begin{lstlisting}[
caption = {make the case that they're both associative reductions, but it's not obvious what the binary operator is from the code for the second one}]
Summation (easy example)

Func out;
out() = 0;
RDom r(0, input.width());
out() = out() + input(r.x);

The complex number with the greatest magnitude, 
and its location (hard example)

Func out;
out() = {0, 0, 0};
RDom r(0, input.width(), 0, input.height());
Expr real = input(r.x, r.y)[0];
Expr imag = input(r.x, r.y)[1];
Expr mag = real * real + imag * imag;
Expr best_mag = out()[0] * out()[0] + out()[1] * out()[1];
Expr c = mag > best_mag;
out() = {select(c, real, out()[0]),
         select(c, imag, out()[1]),
         select(c, r.x, out()[2]),
         select(c, r.y, out()[3])};
\end{lstlisting}

However, Halide did not support parallel or *vectorized* reductions till now without changing algorithm (which fails to deliver core promise of language in which schedule and algorithm should be separate and not affect each other). We present a Halide scheduling primitive (DAG transformation) that creates a new data parallelizable/vectorizable axis out of a reduction. \\

rfactor splits an update (**associative update**) into an intermediate which computes the partial results and a new update definition which merges the partial results -> creates separate copies of the reduction dimension which exposes a new data parallelism (parallelizable + vectorizable), and a second stage that combines those partial results. Combined with other Halide scheduling directives, such as split, this allows Halide to represent a broader class of schedules, including parallel associative reduction. 

<Insert some code snippet of pipeline produced by rfactor, including performance numbers for it -> serial, hand-rolled, using rfactor> \\

Other benefits: code reduction, supports purity/separation of algorithm and schedule, and portability, which is especially important for auto-scheduling \cite{Mullapudi:2016:ASH:2897824.2925952}. \\


\section{Background and Related Work}
Programmers define the \emph{algorithm} in Halide through a Halide \emph{function}, which consists of sequence of stages; these stages are the unit on which scheduling occurs. By default, each stage represents a perfectly-nested loop nest in which a single value of the \emph{function} is computed and stored in the innermost loop per iteration. Stages after the first are called \emph{update} stages, and are allowed to recursively refer to the function. Some of the loops are data parallel and are constrained to be race-condition free by syntactic restrictions. These data-parallel loops iterate over variables called \code{Var}s. The bounds of these loops are inferred by Halide using interval arithmetic.  Other loops may have user-specified bounds and a user-specified nesting order, and fewer syntactic restrictions on their use. These are known as \code{RVar}s (for reduction \code{Vars}), which together define a reduction domain or \code{RDom}. \code{RVar}s are used to express reductions, scattering, scans, etc. Each of these loop types, defined by \code{Var}s and \code{RVar}s, can be manipulated in various ways through Halide scheduling primitives: they can be tiled, unrolled, mutually interchanged, etc., provided that the nesting order of \code{RVar}s is respected. 

While \code{Var}s are safe to parallelize or vectorize by construction -- \code{Var}s represents the naturally data-parallel axes of an \emph{algorithm} -- \code{RVar}s can be parallelized or vectorized if and only if Halide can prove that no race condition exists. This makes parallelizing or vectorizing stages that use only \code{RVar}s difficult. For example, consider the two-dimensional convolutional blur kernel shown in Listing~\ref{lst:blur_loopness}, which is easily parallelizable across \code{Var} $x$ and $y$. The histogram of an image (see Listing~\ref{lst:histogram_loopness}), on the other hand, is harder to parallelize since its update stage only involves \code{RVar}s. In order to parallelize a reduction like histogram, one needs to be able to factorize it into slices that have no dependencies on each other.

Although much prior work has explored automatic generation of parallel associative reductions from a serial reduction, most work assumes an explicit associative binary reduction operator is given, which is not applicable to Halide. Since Halide does not support reduction using a binary operator as a first-class primitive, reductions in Halide are implemented through usage of non-data-parallel \code{RVar}s. For Halide to support parallel reductions, it needs to be able to deduce an equivalent binary associative reduction operator and its identity from a serial reduction expressed as an imperative Halide \emph{update}. 

Prior work by Morita et al.~\cite{Morita:2007:AIG:1250734.1250752} introduced automatic generations of divide-and-conquer parallel programs framework based on the third homomorphism theorem and derivation of weak-right inverse. However, it requires programmers to specify the leftwards and rightwards forms of the sequential function which may not be obvious to derive. Teo et al.~\cite{Teo:1997:DEP:266670.266697} proposed a method to synthesize parallel divide-and-conquer
programs from a recurrence function (which is similar in form to a Halide serial reduction) through induction. They first derive two equivalent pre-parallel forms of the recurrence function by applying some generalization rules and deduce the intermediate and merge reduction functions through induction on those two pre-parallel forms. Although it can be applied to solve some complex recurrences, such as reduction of complex multiplication, the technique requires long derivation time and is unable to deal with reductions like \code{argmin}, which require non-trivial re-ordering of the chain of conditionals during the induction steps. 

Recent work has applied program synthesis, which automatically discovers executable code based on user intent derived from examples or other constraints, to generate parallel programs. Smith et al.~\cite{Smith:2016:MPS:2908080.2908102} used program synthesis to automatically generate MapReduce-style distributed programs from input-output examples. \textsc{Sketch}~\cite{Solar-Lezama:2008:PSS:1714168} and \textsc{Rosette}~\cite{Torlak:2013:GSL:2509578.2509586} are two solver-aided programming languages with support for program synthesis.  MSL~\cite{Xu:2014:MSE:2683593.2683628} is a synthesis-based language for distributed implementations that can derive many details of the distributed implementation from serial specifications.  We discuss \textsc{Sketch} and \textsc{Rosette} in Section~\ref{synthesize}.

Superoptimization~\cite{Granlund:1992:EBU:143095.143146, Massalin:1987:SLS:36206.36194} searches for the shortest or most optimized way to compute a branch-free sequence of instructions, by exhaustively searching over a space of possible programs. These rewrites can then be turned into peephole optimizations in compilers. More recent work has used stochastic search~\cite{Phothilimthana:2016:SUS:2872362.2872387, Schkufza:2013:SS:2490301.2451150} and program synthesis~\cite{Lopes:2015:PCP:2737924.2737965} to find replacements for larger sequences of instructions.
In this work, we find equivalent replacements for a Halide reduction through a combination of enumeration and synthesis; in addition, though our domain is more restricted, we search for larger replacements than most superoptimizers.

\begin{lstlisting}[caption={Convolution blur kernel is easily parallelizable across \code{Var} $x$ adn $y$.}, label={lst:blur_loopness}]
// First stage
for y in range(input.height()):
  for x in range(input.width()):
    blur[x][y] = 0
// Update stage
parallel for y in range(input.height()):
  parallel for x in range(input.width()):
    for ry in range(kernel.height()):
      for rx in range(kernel.width()):    
        blur[x][y] += 
          kernel[rx][ry]*input[x+rx-1][y+ry-1] 
\end{lstlisting}

\begin{lstlisting}[caption={Histogram of an image is hard to parallelize since its update stage does not involve \code{RVar}s.\textbf{SK: it doesn't involve Rvars?}}, label={lst:histogram_loopness}]
// Serial version
// First stage
for x in range(256):
  hist[x] = 0
// Update stage
for ry in range(input.height()):
  for rx in range(input.width()):
    hist[clamp(int(input[rx][ry]), 0, 255)] += 1
\end{lstlisting}


\section{Associative Reduction in Halide}
\subsection{Reductions in Halide}

Serial reductions in Halide (e.g. summation over an array, histogram, etc.) are implemented using \code{RVar}s or \code{RDom}s. An \code{RVar} is an implicit serial loop, and an \code{RDom} is an ordered list of \code{RVar}s specifying a serial loop nest. Since \code{RVar}s are not trivially parallelizable or vectorizable, a programmer must manually \emph{factor} a reduction into an \emph{intermediate} function that performs reduction over distinct slices of the domain, and a \emph{merge} function that combines those partial results.

This manual manipulation is tedious and error-prone, especially when the reduction domain is non-rectangular (see Listing \ref{lst:circular_max_1}). To further complicate matters, it is hard to infer what binary reduction operator is equivalent to a Halide update definition, and even then, many binary operators are not obviously associative (e.g. $x + y + 7xy$ is in fact associative). We will defer these issues to section \ref{synthesize}, and for now assume that given a Halide update definition we can deduce the equivalent associative binary operator and its identity. Note that we do not require the binary operators to be commutative.

\subsection{The \code{rfactor} Transformation}

To remove the burden of \emph{factoring} a reduction from the programmer, we introduce a new scheduling primitive called \code{rfactor}. This splits a reduction into pair of reductions, which we will call the \emph{intermediate} stage and the \emph{merge} stage. \code{rfactor} takes as input a list of \code{<RVar, Var>} pairs. \code{RVar}s not in the list are removed from the \emph{merge} stage and lifted to the \emph{intermediate} stage. The remaining \code{RVar}s become \code{Var}s in the \emph{intermediate} stage, which allows them to be parallelized or vectorized. See Figure \ref{fig:rfactor} for a simple example. The listings below demonstrate more complex usage.

 Note that we limit the scope of \code{rfactor} to reductions where the \emph{intermediate} and \emph{merge} stages have the same equivalent binary associative reduction operator. For instance, if the equivalent binary associative reduction operator of the \emph{intermediate} stage is \code{min(x, y)}, then that of the \emph{merge} stage must also be \code{min(x, y)}. Another restriction is that the binary associative operator must have an identity, as it is used to initialize the \emph{intermediate} stage. Not all associative binary operators have identities (e.g. $2xy$, where $x, y \in \mathds{Z}$).

\begin{lstlisting}[caption={Computing the histogram of a two-dimensional image in Halide. The RDom defines an implicit loop nest over \code{r.x} and \code{r.y}. Halide will not permit either of these loops to be parallelized, as that would introduce a race condition on the += operation.}, label={lst:histogram_rfactor_1}]
// Algorithm
Func hist;
Var i;
hist(i) = 0;
RDom r(0, input.width(), 0, input.height());
hist(input(r.x, r.y)) += 1;

// Schedule
hist.compute_root();
\end{lstlisting}

\begin{lstlisting}[caption={A manually-factored histogram. The programmer has introduced an intermediate function that computes the histogram over each row of the input. This intermediate is data-parallel over y, and so it can be parallelized. The original function \code{hist} now merely sums these partial histograms. It is data-parallel over histogram buckets, and the programmer has vectorized it.}, label={lst:histogram_rfactor_2}]
// Algorithm
Func intm;
Var i, y;
intm(i, y) = 0;
RDom rx(0, input.width());
intm(input(rx, y)) += 1;

Func hist;
hist(i) = 0;
RDom ry(0, input.height());
hist(i) += intm(i, ry);

// Schedule
intm.compute_root().update().parallel(y);
hist.compute_root().update().vectorize(i, 4);
\end{lstlisting}

\begin{lstlisting}[caption={Using \code{rfactor}, the programmer can produce the same machine code as in \ref{lst:histogram_rfactor_2}, using the simpler algorithm in \ref{lst:histogram_rfactor_1}. While the schedule is more complex, recall that it is only the five lines of algorithm that determines correctness. The programmer was able to transform the code to exploit parallelism without risking introducing a correctness bug.}, label={lst:histogram_rfactor_3}]
// Algorithm
Func hist;
Var i;
hist(i) = 0;
RDom r(0, input.width(), 0, input.height());
hist(input(r.x, r.y)) += 1;

// Schedule
Var y;
hist.compute_root()
Func intm = hist.update().rfactor(r.y, y);
intm.compute_root().update().parallel(y);
hist.update().vectorize(i, 4);
\end{lstlisting}

\begin{lstlisting}[caption={Computing the maximum over a circular domain. Reduction domains need not be rectangular. In this case we use \code{RDom::where} to restrict it to the points that lie within a circle of radius 10.}, label={lst:circular_max_1}]
// Algorithm
Func max_val;
max_val() = 0;
RDom r(0, input.width(), 0, input.height());
r.where(r.x*r.x + r.y*r.y <= 100);
max_val() = max(max_val(), input(r.x, r.y));

// Schedule
max_val.compute_root();
\end{lstlisting}

\begin{lstlisting}[caption={Manually factoring this reduction requires also manipulating the predicate associated with the RDom. The identity for \code{max} is the minimum value of the type in question.}, label={lst:circular_max_2}]
// Algorithm
Func intm;
Var y;
intm(y) = input.type().min();
RDom rx(0, input.width());
rx.where(rx*rx + y*y <= 100);
intm(y) = max(intm(y), input(rx, y));

Func max_val;
max_val() = 0;
RDom ry(0, input.height());
max_val() = max(max_val(), intm(ry));

// Schedule
intm.compute_root();
    .update().parallel(y);
max_val.compute_root();
\end{lstlisting}

\begin{lstlisting}[caption={Using \code{rfactor} in the schedule can produce the same machine code from the simpler form of the algorithm.}]
// Algorithm
Func max_val;
max_val() = 0;
RDom r(0, input.width(), 0, input.height());
r.where(r.x*r.x + r.y*r.y <= 100);
max_val() = max(max_val(), input(r.x, r.y));

// Schedule
Var y;
max_val.compute_root();
Func intm = max_val.update().rfactor(r.y, y);
intm.compute_root();
    .update().parallel(y);
\end{lstlisting}




\section{Evaluation}
Performance results using rfactor (overall speedup): 2D argmin, complex multiplication, 2D histogram, dot product, matrix multiply. \\

Synthetic functions (also to show limitations): approximating 128-bit add with 2 64-bit integers -> z3 cannot prove that it is associative, although the max/min forms are provable with z3. \\

Limitations: we need an identity, symmetric intermediate and merge functions: they should be of the same form, constrained by the look-up table (we can only match to whatever are in the table -> whatever z3 can prove to be associative). Some associative ops (e.g. 4x4 matrix multiply) are just expensive to generate. Technically it's doable, provided we limit the ops to addition and multiplication and restricting the variables involved in the expr to be unique (no repeats) during lookup table generation. \\

Real-world stuff: same code that can be simplified when using rfactor \\

Performance of generation/search/synthesis. <TODO: Not sure what this is about? time taken when doing the matching? how many associative operations can be generated within some hours? > \\

Case study of importance of "code reduction"? <TODO: Not sure what this is about? > \\ 

\section{Discussion}
\input{discussion}

\bibliographystyle{abbrv}
\bibliography{sigproc}  % sigproc.bib is the name of the Bibliography in this case
% You must have a proper ".bib" file
%  and remember to run:
% latex bibtex latex latex
% to resolve all references
%
% ACM needs 'a single self-contained file'!
%
\end{document}
